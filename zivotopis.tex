\renewcommand{\leftmark}{Životopis}
\chapter*{Životopis}
\addcontentsline{toc}{chapter}{Životopis}

% Životopis autora doktorskog rada treba biti napisan u trećem licu jednine, a
% opsegom ne smije prelaziti 1500 znakova (uključujući razmake).

Filip Boltužić rođen je 19. rujna 1988. u Sisku u Hrvatskoj. Prediplomski
studij računarstva završio je 2010. godine na Fakultetu elektrotehnike i računarstva
Sveučilišta u Zagrebu s temom ``Primjena algoritma kolonije pčela na kombinatoričke probleme''. 
Na istome je fakultetu završio i diplomski studij računarstva
(smjer računarska znanost) s temom ``Tehnike prikupljanja i vizualizacije
velikih skupova podataka''. 

Od listopada 2012. do travnja 2014. zaposlen je u odjelu Poslovne inteligencije 
Zagrebačke banke Unicreditgroup d.o.o. kao 
mlađi analitičar. Od srpnja 2014. do rujna 2017. zaposlen je 
u Amazon Web Services Ireland Ltd. kao razvojni inženjer. 
Od siječnja 2018. do siječnja 2020. 
zaposlen je na Zavodu za elektroniku, mikroelektroniku, računalne
i inteligentne sustave Fakulteta elektrotehnike i računarstva  
na projektu Uspostava integralnog sustava za 
upravljanje službenom dokumentacijom Republike Hrvatske kao voditelj razvoja.

Njegovi istraživački interesi obuhvaćaju područje obrade prirodnog jezika,
pretraživanja informacija i strojnog učenja. 
Član je strukovne udruge ACL (Association for Computational Linguistics). 
Govori engleski jezik.

\section*{Popis objavljenih djela}

% \subsection*{Rad u časopisima}
% 
% % TODO make enumerate
% 
% \begin{itemize}
% % \item Prezime1, InicijalImena1., Prezime2, InicijalImena2., Prezime3,
% % 	InicijalImena3., ``Naslov članka'', Naziv časopisa, Vol. X, No. Y (ili
% % 		Issue Y), mjesec i godina, str. A-B.
% \item Boltužić, F., Di Buono, M.P., Šnajder, J., Semantic-web journal
% \end{itemize}

\subsection*{Radovi na međunarodnim znanstvenim skupovima}

\begin{itemize}

\item Šoštarić, M., Pavlović, N., Boltužić, F., 
``Domain Adaptation for Machine Translation Involving a Low-Resource Language:
Google AutoML vs. from-scratch NMT Systems'',
Proceedings of the 
41st edition of Translating and the Computer Conference (TC41)
of The International Association for Advancement in Language Technology
(AsLing), studeni 2019.
\item Brassard, A., Kuculo, T., Boltužić, F., Šnajder, J., 
``TakeLab at SemEval-2018 Task12: Argument Reasoning Comprehension with Skip-Thought Vectors'',
12th International Workshop on Semantic Evaluation (SemEval-2018),
siječanj 2018., str. 1133-1136
\item Boltužić, F., Šnajder, J., 
``Toward Stance Classification Based on Claim Microstructures'',
Proceedings of the 8th Workshop on Computational Approaches to Subjectivity,
Sentiment and Social Media Analysis (WASSA) in conjunction with 
Conference on Empirical Methods in Natural Language Processing (EMNLP 2017),
rujan 2017., str. 74-80
\item Boltužić, F., Šnajder, J., 
``Fill the Gap! Analyzing Implicit Premises between Claims from Online Debates'',
Proceedings of the 3rd Workshop on Argument Mining in conjunction 
with 54th Annual Meeting of the Association for Computational Linguistics (ACL 2016),
lipanj 2016., str. 124-133
\item Tutek M., Sekulić I., Gombar P., Paljak I., Čulinović F., Boltužić
F., Karan M., Alagić D., and Šnajder J., ``TakeLab at
SemEval-2016 Task 6: Stance Classification in Tweets Using a
Genetic Algorithm Based Ensemble.'', Tenth International
Workshop on Semantic Evaluation (SemEval), lipanj 2016., str. 464–468
\item Boltužić, F., Šnajder, J.,
``Identifying prominent arguments in online debates using semantic textual similarity.'',
Proceedings of the 2nd Workshop on Argumentation Mining in conjunction
with 17th Annual Conference of the North American Chapter of the Association for
Computational Linguistics: Human Language Technologies
(NAACL-HLT 2019), lipanj 2015., str. 110-115
\item Boltužić, F., Šnajder, J., 
``Back up your Stance: Recognizing Arguments in Online Discussions'', 
Proceedings of the First Workshop on Argumentation Mining 
in conjunction with 52st Annual Meeting of the Association for Computational Linguistics (ACL 2014),
lipanj 2014., str. 49-58
\end{itemize}
