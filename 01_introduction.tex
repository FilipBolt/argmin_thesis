\chapter{Introduction}

- we argue every day; \\
- Whether it is a domestic discussion which color the bathroom should 
be painted with, a political TV show debate on how will the latest tax increase 
impact small businesses, or a discussion at work on which text editor makes
code editing the fastest, arguments are used to convince the other discussion 
participant(s) to adhere to a single opinion. \\
- More formally, a dialogue or a conversation is defined as a goal-directed
conventional framework in which two partners reason 
together in an orderly way, 
according to the rules of politeness or normal exchange expectations
of cooperative argumentation for the type of exchange they are
engaged in \citep{walton1998new} \\
- from a cooperative and productive argumentative discussion, 
an informed, critically evaluated decision can be made \\
- being able to make important decisions only further 
underlines the importance of argumenation \\
- understanding public opinion on controversial topics, such as 
\textit{marijuana legalization}, \textit{gay marriage legalization}, 
\textit{euthanazia legalization}, and many more is important for
policy making \\
- knowing merely stance (whether someone is pro or con on the topic)
is only surface level information, as arguments behind that stance are
crucial to understance where that stance comes from \\
- analyzing argumentation boils down to understanding and logically connecting
claims into arguments \\
% TODO add claim definition from argonotlogy paper
% - we wish to explore computational methods of 
- we wish to computationally ease the process of analyzing arguments for a 
specific discussion topic \\

%- defining argumentation goes back to Aristotle \\
%- interest in argumentation started with computational argumentation, 
%roughly with \citep{dung1995acceptability} \\

\noindent - computational analysis of arguments started with formalizing arguments, more
specifically, the area of computational argumentation \\
- field of computational argumentation developed formal, logic-based 
accounts of arguments \\
- argumentation is formed in language and therefore related to the field
of computational linguistics \\
- argumentation research has mostly relied on knowledge and 
logic-based solutions, whereas computational lingustics has adopted a 
more , especially with the advent of machine learning \\
- scalability issues limit the application reasoning, logic based
approaches, whereas data-driven approaches often yield non-logical
solutions \\
- one argument towards logic-based approaches 
is that a well-known problem such as POS tagging seems to have reached its
apex with 97\% performance and that rule-based approaches may improve it 
\citep{manning2011part} \\
- we wish to combine logic, reasoning-basec approaches with data-driven ones \\

\noindent - our goal is to allow for argument analysis from raw text \\
- the text is expected to be from online discussions, more specifically internet 
discussions, which carry have their own set of rules and conventions \\
- we wish to work on problems to extract claims, group claims into arguments, and analyze 
claims, potentially deriving new claims (premises) \\

% - in this work we wish to balance between formalized and unformalized approaches \\
% - we aim to use non-formalized approaches to identify and extract claims from text, then use
% formalized and structured approaches to derive arguments from claims along with their
% underlying premises of arguments \\

\section{Argumentation Mining of Claims}

- argument analysis is performed from many angles \\
- computational argumentation starts with predefined claims and connects them 
with (support or attack) relations to form arguments and networks of arguments \\
- these argument networks (graphs) are then usually analyzed in terms of acceptability \\
- however, this approach can be rather expensive, as no automatic methods currently exist
to determine claims \\
- argumentation mining relies on natural language processing techniques to
extract claims from text, recognize relations between claims and then analyze them
by means of clustering on deriving extra claims (or premises) from 
extracted ones \\
- we wish to walk the line between argumentation mining and computational argumentation 
by extracting claims by means of natural language processing, after which we 
experiment with both structured and unstructured approaches of claim analysis \\

\section{Contributions}

% TODO this is from the javni razgovor document
The research aims to improve the state of
the art in argumentation mining by means of claim structuring and the
development of a computer system prototype for claim analysis, which would
allow for better understanding of argumentation in internet discussions. 
The prospective original scientific contribution consists of: 
\begin{enumerate}
\item A method for modeling of argumentation in internet discussions using an
two-level ontology, where the first level contains topic-specific knowledge,
while the second level models the claim patterns;
\item A computational method for detection and
structuring of claim in argumentative discourse based on supervised machine
learning;
\item A framework and a prototype of a system for computer-aided
analysis of claims in internet discussions which links together the detection,
structuring, and the analysis of claims.  
\end{enumerate}

\section{Thesis structure}

The thesis is structured as follows. Finally, chapter~\numberstringnum{\getrefnumber{chap:conclusion}}
concludes the thesis and gives directions for future work. 
