\chapter{Introduction}

We argue every day; whether it is a domestic discussion on which color the
bathroom should be painted with, a political TV show debate discussing how will
the latest tax increase impact small businesses, or an office discussion about
which text editor makes code editing the fastest, arguments are used to
convince the other discussion participant(s) to adhere to a single opinion.
More formally, a dialogue or a conversation is defined as a goal-directed
conventional framework in which two partners reason 
together in an orderly way, 
according to the rules of politeness or normal exchange expectations
of cooperative argumentation for the type of exchange they are
engaged in \citep{walton1998new}. 
From a cooperative and productive argumentative discussion, 
an informed, critically evaluated decision can be made. 
Being able to make important decisions only further 
underlines the importance of argumentation. 
Understanding public opinion on controversial topics, such as 
``\textit{marijuana legalization}'', ``\textit{gay rights}'', 
``\textit{abortion abolition}'', and many more, is important for
policy making. 
With respect to the their position towards a topic, 
discussion participants are usually divided into two groups: 
proponents (pro) and opponents (con). The position towards a topic is denoted as
\emph{stance}. But, knowing only the stance of a person
is only surface level information, as arguments behind that stance are
crucial to understand where that stance comes from. Arguments
are formed by logically connecting \emph{claims} -- the building blocks of argumentation. 
Claims represent facts and
individual beliefs within a specific domain. When arguing, claims are backed up
with other claims -- justifications. For example, the sentence ``\emph{my
argument states that abortion is morally wrong because it deprives the fetus of
a valuable future}'' has two claims ``\emph{abortion is morally wrong}'' and,
``\emph{because it deprives the fetus of a valuable future}'' which pinpoint
the speakers' stance.  To determine claims behind a certain  stance, it is
often required to go beyond written text. 

Reading between the lines is a skill humans naturally attain mainly through
reading experience and social interaction. While reading internet discussion
comments, in an effort to understand the intent of a writer, the reader is
often biased by numerous small, albeit important, cues, such as the nickname
of the speaker, (speaker named \emph{MarleyForever} is likely to be pro
``\emph{marijuana legalization}'') or the groups that speaker belongs to
(speaker belonging to \emph{vegetarian tips}
 is more likely to be con ``\emph{gun legalization}'').
All these biases heavily influence how the reader perceives the speakers' comment. 
By doing so, the reader infers additional statements that have to hold
in order to make the argumentation of the comment logically consistent. 
Determining stance and arguments behind the stance thus require 
fully understanding the context in which opinion of the speaker
is made, as well as implicit information which the speaker 
(intentionally or unintentionally) omits. 
Before analyzing argumentation (and acting on it), it is essential to
understand stance, arguments, and underlying implicit assumptions of the speaker. 

The amount of internet discussions is growing rapidly, particularly in social
media, as 65\% of adults in the US use social media \citep{perrin2015social}. 
To handle the growth of argumentation content, two research areas have profiled
thus far: \textit{computational argumentation} and \emph{argumentation mining}.
Computational argumentation uses formal, logic-based accounts of arguments. In
computational argumentation users input formalized arguments in order to
perform various argumentation analysis, such as deriving the set of accepted
(logically consistent) arguments.
On the other hand, argumentation mining works with raw text.  It is founded on
the premise that argumentation is formed in language, therefore it adopts a
computational linguistics approach. 
Argumentation mining usually employs natural language processing methods to extract
argumentation from text. 

In this thesis, we combine reasoning-based approaches from
computational argumentation and natural language processing methods
from argumentation mining to define a framework for the 
argumentative analysis of claims in internet discussions. 
The goal of the framework is to provide argumentation analysts 
an overview of the prominent claims upon a set of discussion comments. 

% - in this work we wish to balance between formalized and unformalized approaches \\
% - we aim to use non-formalized approaches to identify and extract claims from text, then use
% formalized and structured approaches to derive arguments from claims along with their
% underlying premises of arguments \\

% \section{Argumentation Mining of Claims}
% 
% - argument analysis is performed from many angles \\
% - computational argumentation starts with predefined claims and connects them 
% with (support or attack) relations to form arguments and networks of arguments \\
% - these argument networks (graphs) are then usually analyzed in terms of acceptability \\
% - however, this approach can be rather expensive, as no automatic methods currently exist
% to determine claims \\
% - argumentation mining relies on natural language processing techniques to
% extract claims from text, recognize relations between claims and then analyze them
% by means of clustering on deriving extra claims (or premises) from 
% extracted ones \\
% - we wish to walk the line between argumentation mining and computational argumentation 
% by extracting claims by means of natural language processing, after which we 
% experiment with both structured and unstructured approaches of claim analysis \\

\section{Contributions}

The research aims to explore structured approaches to
argumentation mining by means of claim structuring and the
development of a computer system prototype for claim analysis, which would
allow for better understanding of argumentation in internet discussions. 
The prospective original scientific contribution consists of: 
\begin{enumerate}
\item A method for modeling of argumentation in internet discussions using an
two-level ontology, where the first level contains topic-specific knowledge,
while the second level models the claim patterns;
\item A computational method for detection and
structuring of claim in argumentative discourse based on supervised machine
learning;
\item A framework and a prototype of a system for computer-aided
analysis of claims in internet discussions which links together the detection,
structuring, and the analysis of claims.  
\end{enumerate}

\section{Thesis structure}

The thesis is divided into four parts. In part~\ref{part:background}, 
we position the research work of this thesis within related work in 
argumentation mining (chapter~\ref{chap:argmin}). We briefly introduce
main methods and tools that will be used throughout the thesis in 
chapter~\ref{chap:methods}. Part~\ref{part:unstruc} describes three 
pre-research studies in which unstructured approaches to 
claim clustering (chapter~\ref{chap:argclu}), 
prominent claim identification (chapter~\ref{chap:argrec}), and 
deriving implicit claims (chapter~\ref{chap:deriving_implicit}) are presented. 
In part~\ref{part:struc} we then explore structured approaches to argumentation mining. 
First, we demonstrate claim segmentation in chapter~\ref{chap:claim_segmentation}.
Then, we propose the microstructure and ontology-based formalizations 
in chapter~\ref{chap:formalization}. Deriving formalizations from 
claim segments automatically is explored in chapter~\ref{chap:claim_structuring}.
To showcase some of the advantages of structured argumentation mining of claims,
we present example argumentation analysis in chapter~\ref{chap:analysis}.
We conclude and provide future work directions in chapter~\ref{chap:conclusion}.
Appendices with datasets produced as part of this thesis 
and their respective annotation
guidelines are listed in part~\ref{part:appendix}. 

% Finally,
% chapter~\numberstringnum{\getrefnumber{chap:conclusion}} concludes the thesis
% and gives directions for future work. 
