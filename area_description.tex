\chapter{Computational Argumentation}

- how computational argumentation started \\
- what stemmed from computational argumentation \\
- from claims to argumentation schemes \\

\chapter{Argumentation Mining}

- Argument Mining is the automatic identification and extraction of the structure
of inference and reasoning expressed as arguments presented in natural language
\citep{lawrence2019argument} \\
- applications of argumentation mining \\
are found in \textit{decision support in medicine}, multi-agent systems, engineering, 
policy decision making \citep{tremblay2016value, byron2019evaluating} \\

\noindent - tasks of argumentation mining are 
\begin{itemize}
\item prominent claim identification,
\item claim clustering,
\item claim extraction,
\item deriving implied claims.
\end{itemize}

\noindent - manual argument analysis \citep{lawrence2019argument}
- manual analysis often involves using various tools to analyze arguments, 
such as Araucaria \citep{reed2004araucaria}, OVA \citep{reed2014ova+}, or
Carneades \citep{gordon2007carneades} \\
- these tools require already extracted claims that the tool users can then 
connect, determine their roles (premise, conclusion) and define 
relationships (attack, support), and even form refined structured,
such as argumentation schemes \citep{walton2008argumentation} \\
Generally, steps to perform argument analysis are:
\begin{itemize}
\item text segmentation,
\item argument / non-argument,
\item simple structure,
\item refined structure.
\end{itemize}

\section{Text segmentation}

- text segmentation involves extraction of fragments of text that form
constituent parts of a argument building block \\
- there have been various definitions of argument building blocks \\
- one strand of research defines them as \textit{Elementary Discourse Units} EDU \\
- some define EDUs as clauses \citep{winter1982towards, givon1983topic}, 
others as sentences \citep{polanyi1996linguistic}, some as prosodic units
\citep{sacks1978simplest} \\
- they all agree that they are non-overlapping atomic units \\
- another strand of research 
 define \textit{Argumentative Discourse Units} (ADU) 
as minimal units of discourse \citep{peldszus2013argument} 
which can be composed of multiple EDUs, also overlapping \\
- as noted in \citep{lawrence2019argument}, there are multiple issues 
with defining the rules to constituete an argument building block, 
which arise particulary when dealing with reported speech \\
- to overcome this, we therefore, define \textit{claims} to allow
for overlapping and don't define them in terms of phoentics or sytnax, but
state they should be semantically atomic \\ 
- an example why we allow overlapping claims: \\
\begin{mydef}
The church stated that people should not perform abortion or 
smoke marijuana. 
\end{mydef} 
- as text segments \textit{the church stated that people should not perform
abortion} and \textit{the church stated that people should not smoke marijuana}
both convey atomic thought, we feel it would be wrong to make this a single 
claim \\
- on the other hand, having it two claims, but omitting \textit{the church stated}
in one claim, but not the other makes a claim completely different as
context is lost \\
- we also don't wish to rely on text containing subjects or predictes as 
texts in online discusses are often times ungrammatical \\
- this means that we consider that utterances
\textit{Yes, definitely!} and \textit{Marijuana should definitely be allowed}
can both claims of the same meaning, given the context they are made in. \\

\section{Argumentative / Non-Argumentative}

\section{Claim clustering}

- Some claims are more prominent than others in terms of frequency.  \\
- Those claims are usually used for claim analysis when trying to recognize 
prominent claims \\
Claim clustering attempts to determine prominent arguments  
by clustering claims. \\

\section{Prominent claim identification}

- describe what prominent claim identification is and why is it useful \\
- example which shows some prominent arguments \\
- List all papers that do prominent claim identification \\
