\section{Prominent claim identification annotation}
\label{sec:argrec_annotation}

Prominent claim identification is the task of recognizing which (from a set of
predefined) prominent claims is mentioned in a target comment and how. The goal
of the task is to label prominent claim-comment pairs with a label:
\begin{itemize}
	\item \textbf{A} -- explicitly attacks the prominent claim
	\item \textbf{a} -- vaguely/implicitly attacks the prominent claim
	\item \textbf{N} -- makes no use of the prominent claim
	\item \textbf{s} -- vaguely/implicitly supports the prominent claim
	\item \textbf{S} -- explicitly supports the prominent claim
\end{itemize}
Labeling the dataset according the guidelines below
produced the \ComArg dataset (described in section~\ref{sec:comarg}.

\subsection*{Annotation Guidelines}

There is an online discussion about ``gay rights''. The topic is ``SHOULD GAY
PEOPLE BE ALLOWED TO MARRY?''. Users post comments on the discussion forum,
which express opinions that are either for or against gay marriages. For
example: COMMENT: ``Gay people shouldn’t marry because they can’t have
children.''
In these discussions, users often refer to certain well-established arguments .
For example, typical arguments are:
\begin{itemize}
\item ARGUMENT 1: ``All people should be treated equally.''
\item ARGUMENT 2: ``Marriage should be between two believers who can produce godly offspring.''
\item ARGUMENT 3: ``Marriage is about more than procreation, therefore gay couples should not be denied the right to marry due to their biology.''
\end{itemize}

\noindent Your task is to detect WHICH arguments are used in a comment and HOW. You will
be presented with three arguments for each comment. For each argument, there
will be five check boxes, numbered 1 (DENIED) through 5 (APPROVED). 

\noindent For each argument, you need to do the following:
\begin{itemize}
\item if the comment directly denies the argument, check 1 (DENIED).
\item if the comment does not refer to the argument, check 3 (NOT MENTIONED)
\item if the comment approves the argument to make its point, check 5 (APPROVED).
\end{itemize}

\noindent You might feel that the comment approves or denies the argument, but you're not
completely certain. If you think that the argument is indirectly or partially
denied, check option 2, which is the option between DENIED and NOT MENTIONED.
Conversely, if you think that the argument is indirectly or partially approved,
check option 4, which is the option between NOT MENTIONED and APPROVED.
Note that it will often be the case that an argument is not mentioned.

Also note that if a comment and an argument express a different opinion, it
does not automatically mean that the argument is denied. For example, a comment
``Marriage is a religious institution, and the major world religions frown upon
homosexuality'' does not deny the argument ``Gay couples should be able to take
advantage of the fiscal and legal benefits of marriage''. Here, the comment and
the argument do express different opinions over the issue, but the argument
itself is not mentioned in the comment.

Consider again the example above. To detect whether and how ARGUMENTS 1-3 are
used in COMMENT, think in the following way.
In case of ARGUMENT 1: ``All people should be treated equally.''

\begin{enumerate}
\item    ARGUMENT 1 refers to human equality
\item    COMMENT does NOT refer to human equality
\item    Therefore, check option 3 (NOT MENTIONED) for ARGUMENT 1.
\end{enumerate}

\noindent In case of ARGUMENT 2: ``Marriage should be between two believers who can
produce godly offspring.''

\begin{enumerate}
\item ARGUMENT 2 implies that having children is a necessary condition for
marriage
\item COMMENT states that gay people shouldn’t marry because they cannot have
children, implying that having children is a necessary condition for marriage
\item BOTH claims are about having children and make the same point
\item therefore, COMMENT supports ARGUMENT 2
\item check option 5 (APPROVED)
\end{enumerate}

\noindent In case of ARGUMENT 3: ``Marriage is about more than procreation, therefore gay
couples should not be denied the right to marry due to their biology.''

\begin{enumerate}
\item ARGUMENT 3 implies that having children is not a necessary condition for marriage
\item COMMENT states that gay people shouldn’t marry because they cannot have children, implying that having children is a necessary condition for marriage
\item BOTH claims are about ``having children'', but COMMENT states the opposite of ARGUMENT 3
\item therefore, COMMENT denies ARGUMENT 3
\item check option 1 (DENIED)
\end{enumerate}
