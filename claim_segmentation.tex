\chapter{Claim segmentation}

% TODO give intro to motivate structure approach
% outline steps of structured approach

%TODO give intro to motivate segmenting claims in general

\section{Data}

\setlength{\tabcolsep}{4pt}

\begin{table*}[t]
\begin{center}
{\footnotesize
	\begin{tabular}{@{}p{0.2\linewidth} p{0.30\linewidth} p{0.30\linewidth} }
\toprule
\textbf{User post} & \textbf{Claim segment} & \textbf{Claim paraphrase}   \\
\midrule
\multirow{3}{*}{\parbox{3cm}{
		\emph{Men should fall in love with women that's why they where created and women should 
		get married to men because it makes everything easier. }
}}
&  
\emph{Men should fall in love with women.}
& \emph{People of opposite sex should fall in love.}
\\
\cmidrule{2-3}
& \emph{that's why they where created} & \emph{Men and women are created to pair.}
 \\
\cmidrule{2-3}
& \emph{women should get married to men because it makes everything easier.} & 
 \emph{Heterosexual marriages make everything easier.}
 \\
 \bottomrule
\end{tabular}}
\end{center}
\caption{An example of a user post segmented into three claim segments with their correspoding paraphrase.}
\label{tab:claim_seg_post_segments}
\end{table*}

- we adopt the dataset of \citet{hasan2014you} which contains 
user posts from online two-sided discussions on a number of issues \\
- for reasons of feasiblity, we consider two topics: ``Gay rights'' and 
``Marijuana'' \\
- we sampled 100 posts (50 \textbf{pro} and 50 \textbf{con}) \\
- first, annotators segment out claims from user posts 
% TODO consider where to put paraphrasing, maybe to go with structuring
- second, after segmenting out a claim, the annotators provide a paraphrased
version of the claim \\
- we assumed that paraphrasing helps understanding of claims \\
- our work is similar to \citep{wyner2016working} who use a controlled language 
for paraphrasing claims \\

- claim segmenting separates argumenative from non-argumentative content \\
- there are many ways a post can be segmented into claims \\
- there is a number of ways to paraphrase a claim \\
- the ambiguity can be reduced by doing these two tasks jointly \\
- the end result paraphrased should be \emph{simplyfing paraphrases} \\
- paraphrases that provide the essence of claims devoid of 
superflous words and phrases \\
- we adopt nine paraphrasing principles: 
\begin{enumdescript}
\item[Argumentativeness] --- Only argumentative text should paraphrased;
\item[Atomicity] --- A claim should convey a single thought; 
\item[Authority] --- Experts in claims from expert opinion should be made explicit in the
paraphrase; 
\item[Brevity] --- Paraphrases should keep only the relevant argumentative content; 
\item[Canonicity] --- Canonical terms and phrases are preffered over idiomatic language; 
\item[Contextuality] --- Claims should be paraphrased by considering their local and topical
context as well as their context; 
\item[Declarativity] --- paraphrases should be in declarative form; 
\item[Dereferencing] --- Pronouns and nominal references should be resolved; and
\item[Explicitness] --- Only explicitly stated information should be paraphrased, and not
whatever might be implied by the claim 
\end{enumdescript}
- we allow for overlapping and discontiguous segments \\
- full annotation guidelines are in appendix~\ref{sec:argseg_annotation} \\
- the annotation for ``Gay Rights'' was carried out by one trained annotator and took 25 hours\\
- the annotation for ``Marijuana'' was carried out by three annotators \\
% TODO add stats time for marijuana 
- the 100 user posts yield 920 claim segments for ``Gay rights'' \\
- table~\ref{tab:claim_seg_post_segments} gives an example \\
- for the ``Gay Rights'' topic, the segments covered 79.6\% of the text, while the remaining
20.4\% may be considered non-argumentative \\
%TODO add stats for the Marijuana topic

\section{Models}

\subsection{Na\"ive Heuristic}

\subsection{Supervised Classification}

\subsection{BiLSTM + CRF Model }


\section{Discussion and Conclusion}
