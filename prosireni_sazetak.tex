\section*{Sažetak}

\subsection*{Računalni postupci dubinske argumentativne analize tvrdnji u internetskim raspravama}

Rad se bavi nizom zadataka iz područja dubinske analize argumentacije. 
Dubinska analiza argumentacija odnosi se na računalnu ekstrakciju argumentacije
iz teksta. S porastom uporabe interneta sve se više argumentacije nalazi na
internetskim raspravama. Osnovni gradivni elementi argumentacije jesu tvrdnje.
Predmet istraživanja jesu računalne metode za dubinsku analizu argumentativnih
tvrdnji u okviru tematski specifičnih internetskih rasprava. Postupak je
zasnovan na modeliranju i strukturiranju tvrdnji pomoću računalnih ontologija,
kojima se opisuju tematski specifični koncepti te generički obrasci
tvrdnji. Strukturiranje tvrdnji pomoću ontologija umanjuje problem
jezične varijabilnosti i implicitnosti tvrdnji, čime se omogućava
učinkovitija računalna analiza argumentacije. Predlažu se metode
nadziranog strojnog učenja za prepoznavanje i strukturiranje tvrdnji
iz tekstova internetskih rasprava te model za analizu argumentacije i
implicitnih tvrdnji svih sudionika rasprave.

\vspace{1cm}
\textbf{Ključne riječi}:  
dubinska analiza argumentacije, obrada prirodnog jezika,
formalno predstavljanje znanje, 
strukturirano predviđanje
