\section*{Sažetak}

\subsection*{Računalni postupci dubinske argumentativne analize tvrdnji u internetskim raspravama}

Rad se bavi nizom zadataka iz područja dubinske analize argumentacije. 
Strukturiranje argumentativnog teksta 
preduvjet je za kvalitetnu analizu argumentacije. 
Potreba za analizom argumentativnog teksta prisutna je u raznim djelatnostima, 
kao što su sažimanje stajališta znanstvenih radova, 
donošenje političkih odluka temeljem javnog mišljenja, 
podučavanje stranog jezika kroz razvoj kritičkog razmišljanja 
i sl. S porastom uporabe interneta sve se više argumentacije nalazi u
internetskim raspravama. Argumentacijom se obrazlaže mišljenje naspram
određene teme. Gradivni elementi argumentacije su argumenti, koji se pak sastoje
od međusobno povezanih tvrdnji. Detekcija i analiza tvrdnji glavni su problemi kojim se
bavi ovaj rad. 

Cilj istraživanja je razvoj rješenja niza zadataka koji su ključni za dubinsku
argumentativnu analizu tvrdnji u internetskim raspravama.  Zadatci uključuju
ekstrakciju i strukturiranje tvrdnji. Rješavanje ovih zadataka vrlo je složeno
zbog višeznačnosti jezika i implicitnog znanja ovisnog o kontekstu iznošenja
tvrdnji. Pri istraživanju, poseban je naglasak bio na izradi radnog okvira za
tematski specifičnu dubinsku argumentativnu rasprave.  Prvo su provedena tri
predistraživanja zasnovana na nestrukturiranim metodama dubinske argumentativne
analize. Iz predistraživanja detektirani su nedostaci nestrukturiranih metoda,
stoga je predložen pristup strukturiranja tvrdnji iskazanih u tekstu, koji se
smatra najvažnijim doprinosom rada. U konačnici se pokazuju 
prednosti korištenja pristupa strukturiranja tvrdnji na primjerima interentske rasprave
na temu legalizacije marihuane. 

Prvi istražen zadatak u sklopu predistraživanja jest pronalazak istaknutih
tvrdnji.  Potrebno je, koristeći komentare korisnika internetskih rasprava (na
temama o legalizaciji homoseksualnih brakova, legalizaciji marihuane,
legalizaciji pobačaja te o pozitivnim i negativnim stranama mandata
predsjednika SAD-a Baracka Obame) pronaći istaknute tvrdnje kojima se sudionici
rasprave najčešće služe.  Prvo se komentari s internetskih rasprava grupiraju u
skupine koje bi trebale sadržavati istovjetne tvrdnje u kojima je centroid
skupine pretpostavljen za istaknutu tvrdnju. S tehničkog aspekta, grupiranje
komentara odvija se na način da su komentari prvo projecirani u vektorski
prostor distribuiranih semantičkih reprezentacija, nakon čega dolazi do
hijerahijskog aglomerativnog grupiranja vektorskih reprezentacija komentara.
Rješavanje ovog zadatka olakšava sažimanje rasprave.  Rezultati grupiranja
pokazali su kako je moguće grupirati komentare u skupine, s time da je najteže
razlikovati tvrdnje potpuno suprotnog stava. Distribuirane
semantičke reprezentacije tvrdnji suprotnog stava često su vrlo slične, 
što otežava pronalazak istaknutih tvrdnji.

Kako bi se istražile metode za bolje razlikovanje tvrdnji (suprotnog stava),
drugi zadatak dubinske argumentativne analize istražen u sklopu
predistraživanja jest postupak prepoznavanja istaknutih tvrdnji u komentarima
internetskih rasprava.  Definirano je da komentari mogu podupirati, pobijati,
ili biti neutralni naspram istaknute tvrdnje. Dodatno, određeno je kako
komentar može eksplicitno ili implicitno podupirati ili pobijati tvrdnju. Prema
tome, svaki komentar odnosi se prema tvrdnji prema jednome od pet prethodno
nabrojanih načina (eksplicitno podupiranje, implicitno podupiranje,
neutralnost, implicitno pobijanje, eksplicitno pobijanje).  
Prikupljen i označen je skup podataka \ComArg koji se sastoji od komentara korisnika
na internetskim raspravama (na teme legalizacije homoseksualnih brakova i
uključivanju riječi ,,tako mi Bog pomogao'' prilikom izricanja zakletve) te
istaknutih tvrdnji na internetskim raspravama s ekvivalentnim temama.  Prepoznavanje istaknute
tvrdnje svedeno je na detekciju odnosa između komentara i tvrdnje.  Predložen
je postupak za prepoznavanje istaknutih tvrdnji temeljen na nadziranom strojnom
učenju. Kao ulaz za model strojnog učenja, korištene su tri skupine značajki: 
\begin{enumerate*}
	\item izlazi  \emph{TakeLab Semantic Textual Similarity} 
		modela semantičke sličnosti teksta,
	\item izlazi \emph{Excitement Open Platform} modela za 
		određivanje logičke posljedice, te
	\item značajka usklađenosti stavova.
\end{enumerate*} 
Provedeno je nekoliko vrsta eksperimenata gdje je evaluirana korisnost
pojedinih kombinacija skupina značajki, prema kojima su se izlazi modela
\emph{Excitement Open Platform} pokazali najkorisnijom značajkom.
Naposlijetku, proveden je eksperiment u kojem je ispitana mogućnost primjene
modela treniranog na jednoj domeni na podatke iz druge domene. U svim
eksperimentima primjećeno je kako detekcija odnosa između komentara i tvrdnje
značajno varira po težini, stoga bi bilo dobro dodatno istražiti odnose između
tvrdnji.  

Naposljetku, treći zadatak u sklopu predistraživanja jest 
pronalazak implicitnih informacija u komentarima internetskih rasprava. 
Implicitne informacije između istaknute tvrdnje i komentara koji podupire ili pobija
tu istaknutu tvrdnju definirane su putem niza tvrdnji koje upotpunjavaju 
lanac logičkog zaključivanja između komentara i istaknute tvrdnje. 
Prvo je prikupljen korpus podataka koji sadrži skupove implicitnih tvrdnji za
parove zadanih istaknutih tvrdnji i komentara internetskih rasprava. 
Skupove implicitih tvrdnji generirala su tri označivača te je ispitano slaganje
između njihovih oznaka. Generirani skupovi implicitnih tvrdnji su
kvalitativno i kvantitativno evaluirani, te je utvrđeno kako je, 
u prosjeku, potrebno tri implicitne tvrdnje kako bi se logički povezalo
komentar s istaknutom tvrdnjom. 
Zatim su predložene metode za prepoznavanje istaknutih 
tvrdnji u komentarima uz korištenje generiranih skupova implicitnih tvrdnji. 
Pokazano je kako korištenje implicitnih tvrdnji nedvojbeno pospješuje 
rješavanje zadatka detekcije istaknutih tvrdnji iz komentara. 
Potom je osmišljen jednostavan heuristički postupak za dohvaćanje generiranih 
tvrdnji koje su potom
korištene za detekciju istaknutih tvrdnji. No, eksperimenti su pokazali kako je 
dohvaćanje implicitnih tvrdnji iznimno zahtjevan problem, te su uglavnom pogrešne
implicitne tvrdnje dohvaćane, zbog čega su poboljšanja u detekciji istaknutih tvrdnji 
bila znatno umanjena. 
Iz ovog predistraživanja zaključeno je kako je pronalazak implicitnih informacija
bitan za kvalitetnu dubinsku analizu analizu argumentacije. 

Temeljem zaključaka iz predistraživanja, predložen je strukturirani pristup
pronalaska istaknutih tvrdnji kroz radni okvir za strukturiranje tvrdnji.  U
takvome se pristupu definira formalna struktura tvrdnji kako bi se umanjio
problem različitih lingvističkih realizacija u tekstu i omogućilo dohvaćanje
implicitih tvrdnje kombinirajući pravila logike i informacije iz eksplicitno
konstatiranih tvrdnji.  Pristup strukturiranoj dubinskoj argumentativnoj
analizi konceptualno je proveden u tri dijela.  Prvo je predložena metoda za
predviđanje tvrdnji iz komentara, zatim su tvrdnje modelirane i strukturirane
pomoću računalnih ontologija, te je, u konačnici, predložen niz metoda zasnovan
na kombinaciji metoda iz dubokog učenja i strukturiranog predviđanja 
za strukturiranje tvrdnji iz teksta. Nakon
provođenja strukturiranog pristupa dubinskoj argumentativnoj analizi, moguće je
raditi razne argumentativne analize utemeljene na logičkim pravilima proizašlih ih računalnih
ontologija, kao što su analize najčešćih tvrdnji u raspravi ili analiza
zajedničkih tvrdnji skupina sudionika rasprave. 

U problemu previđanja tvrdnji iz komentara potrebno je izdvojiti nedjeljive jedinice
argumentativnog teksta (tvrdnje) u nestrukturiranom tekstu (komentaru).
Problem je definiran sukladno srodnim
problemima označavanja sekvenci, kao što je problem ekstrakcije imenovanih
entiteta. 
Definirano je devet principa kojima se utvrđuje što čini jednu tvrdnju. 
Temeljem definiranih principa, prošireni su korpusi podataka korišteni u 
predistraživanju (tekstovi internetskih rasprava na temama 
legalizacije homoseksualnih brakova i legalizacije marihuane). 
U komentarima internetskih rasprava označene su tvrdnje i njihove parafraze.
Formalno, problem predviđanja tvrdnji kodiran je na dva načina: 
korištenjem višestrukih binarnih oznaka (engl. \emph{multi-label}) 
te uporabom \texttt{BIO} oznaka. 
Za oba načina kodiranja
predloženo je više modela: heuristički postupak zasnovan na razdvajanju po 
interpunkciji, model potpornih vektora, te model zasnovan na kombinaciji dubokog
učenja i strukturiranog previđanja. Predloženi modeli eksperimentano su vrednovani 
te se pokazalo kako model zasnovan na kombinaciji dubokog učenja i strukturiranog predviđanja
pokazuje najbolje performanse. Predloženi modeli postigli 
su adekvatne performanse na temelju čega se može zaključiti kako je 
je moguće kvalitetno detektirati tvrdnje iz komentara. 

S ciljem ublažavanja niza tipičnih problema prilikom obrade
nestrukturiranog teksta, kao što su 
manifestacija različitih lingvističkih realizacija,
višeznačnosti riječi i izraza, te
implicitnosti teksta, 
izdvojene tvrdnje strukturirane su pomoću računalnih ontologija. 
Strukturiranje tvrdnji definirano je 
kroz dvorazinsku ontologiju. 
Gornja razina ontologije modelira generičke obrasce tvrdnji, dok 
domenska (donja) razina ontologije opisuje tematski specifične koncepte. 
Prilikom strukturiranja tvrdnji, prvo je potrebno definirati 
domensku ontologiju te sve popratne tematski specifične koncepte i
svojstva za svaku pojedinu temu rasprave. Zatim je moguće kombinirati 
definirane koncepte s obrascima gornje razine ontologije kako bi se tvrdnje strukturirale. 
Postupak strukturiranja tvrdnji kroz dvorazinsku argumentativnu ontologiju
validiran je na studiji slučaja. Internetska rasprava na temu
legalizacije marihuane odabrana je kako bi se izradila domenska ontologija. 
Izrađena domenska ontologija potom je povezana s gornjom ontologijom.
Korpus tvrdnji kreiran za potrebe predviđanja tvrdnji proširen je
oznakama strukturiranih tvrdnji. Svakoj tvrdnji tri označivača su dodijelila 
po jednu strukturiranu tvrdnju. 
Provedena je analiza njihovih oznaka kako bi se utvrdilo koliko
strukturirane tvrdnje uistinu reduciraju probleme obrade
nestrukturiranog teksta. 

Kako bi se dovršio zadnji korak strukturiranog radnog okvira za dubinsku
argumentativnu analizu predložene su metode za strukturiranje tvrdnji.
Formalno, problem strukturiranja tvrdnji definiran je na dva načina:
predviđanjem čitave strukture odjednom naspram predviđanja komponenti
strukture.  U slučaju predviđanja pojedinih komponenti strukture konačna
strukturiranja tvrdnja dobija se spajanjem rezultata predviđanja pojedinih
komponenti. Osmišljeni su modeli zasnovani na nadziranom strojnom učenju koji
bi automatizirali strukturiranje tvrdnji iz nestrukturiranih tvrdnji: model
potpornih vektora, ulančani klasifikatori (engl.  \emph{chain classification})
gdje je jedan klasifikator model potpornih vektora, skup ulančanih
klasifikatora (engl. \emph{ensemble chain classification}).  Također,
predložena dvosmjerna povratna neuronska mreža koja istodobno detektira i
strukturira tvrdnje iz komentara.  Navedeni modeli su eksperimentalno vrednovani
uspoređujući predviđanje strukture odjednom i po komponentama.  U
eksperimentima je korišten korpus strukturiranih tvrdnji kreiran prilikom
izrade domenske ontologije za temu legalizacije marihuane.    Prema
eksperimentalnom vrednovanju metoda  skupa ulančanih klasifikatora pokazala se
najboljom. Metoda istodobnog predviđanja i strukturiranja tvrdnji nije pokazala
dobre performanse zbog propagiranja pogrešaka u koraku detekcije tvrdnji.
Generalno gledano problem strukturiranja tvrdnji pokazao se kao vrlo težak
problem, uglavnom zbog velikog broja mogućih rješenja, pogotovo u slučaju
predviđanja strukture po komponentama. Potrebno je dodatno istraživanje kako
bi se pronašla dovoljno dobra metoda za automatsko strukturiranje 
tvrdnji nad novim temama i tvrdnjama. 

Kako bi se demonstrirala prednost strukturiranih tvrdnji nad nestrukturiranima
za dubinsku analizu argumentacije napravljen je niz analiza na temi
legalizacije marihuane. Analize uključuju određivanje najučestalijih tvrdnji u
raspravi te najučestalijih tvrdnji vezanih uz specifični (domenski) koncept,
izvođenje implicitnih tvrdnji pravilima domenske ontologije te korištenjem
korisnički definiranih aksioma, analizu stava sudionika kroz vrijeme, te
grupiranje sudionika temeljem zajedničkih tvrdnji u raspravi.  Niz analiza
pokazuje načine za dobivanje implicitnih tvrdnji koristeći strukturirane
tvrdnje. Za implicitne tvrdnje dobivene iz strukturiranih tvrdnji moguće je
dobiti lanac logičkog zaključivanja temeljem kojeg se izvedene dobivene
implicitne tvrdnje. 

Radni okvir za dubinsku argumentativnu analizu tekstova internetskih rasprava
čine model za detekciju tvrdnji, korisnički definirana domenska ontologija integrirana
u dvorazinsku ontologiju tvrdnji, te model za strukturiranje
tvrdnji pomoću dvorazinske ontologije. Koristeći takav radni okvir definiran u sklopu
ovog rada moguće je raditi argumentativne analize nad strukturiranim tvrdnjama nad
proizvoljnom domenom rasprave.

\vspace{1cm}
\textbf{Ključne riječi}:  
dubinska analiza argumentacije, obrada prirodnog jezika,
formalno predstavljanje znanja, 
strojno učenje, strukturirano predviđanje
