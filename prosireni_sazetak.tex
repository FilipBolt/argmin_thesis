\section*{Sažetak}

\subsection*{Računalni postupci dubinske argumentativne analize tvrdnji u internetskim raspravama}

Rad se bavi nizom zadataka iz područja dubinske analize argumentacije. 
Strukturiranje argumentativnog teksta 
preduvjet je za kvalitetnu analizu argumentacije. 
Potreba za analizom argumentativnog teksta prisutna je u raznim djelatnostima, 
kao što su sažimanje stajališta znanstvenih radova, 
donošenje političkih odluka temeljem javnog mišljenja, 
podučavanje stranog jezika kroz razvoj kritičkog razmišljanja 
i sl. S porastom uporabe interneta sve se više argumentacije nalazi u
internetskim raspravama. Argumentacijom se obrazlaže mišljenje naspram
određene teme. Gradivni elementi argumentacije su argumenti, koji se pak sastoje
od međusobno povezanih tvrdnji. 

Cilj istraživanja bio je razvoj rješenja niza zadataka koji su ključni za 
dubinsku argumentativnu analizu tvrdnji u internetskim raspravama. 
Zadatci uključuju ekstrakciju i strukturiranje tvrdnji tvrdnji. Rješavanje
ovih zadataka vrlo je složeno zbog višeznačnosti jezika i implicitnog 
znanja ovisnog o kontekstu. Pri istraživanju poseban je naglasak bio na izradi
radnog okvira za tematski specifičnu dubinsku argumentativnu rasprave. 
Prvo su provedena tri predistraživanja zasnovana na nestrukturiranim 
metodama dubinske argumentativne analize. Iz predistraživanja detektirani su 
nedostaci nestrukturiranih metoda, stoga je predložen pristup strukturiranja tvrdnji
iskazanih u tekstu, 
koji se smatra najvažnijim doprinosom rada. 

Prvi istražen zadatak u sklopu predistraživanja jest pronalazak istaknutih tvrdnji. 
Potrebno je, uz skup komentara s internetske rasprave, 
pronaći istaknute tvrdnje kojima se sudionici rasprave najčešće služe.  
Prvo se komentari s internetskih rasprava grupiraju u grupe
koje sadrže istovjetne tvrdnje, a centroid grupa pretpostavljen je za 
istaknutu tvrdnju. Komentari se hijarhijski grupiraju koristeći 
njihove distributirane semantičke reprezentacije. 
Rješavanje ovog zadatka olakšava sažimanje rasprave.

Drugi zadatak jest postupak prepoznavanja istaknutih tvrdnji u 
komentarima internetskih rasprava. 
Komentari mogu biti u podupirati ili pobijati istaknute tvrdnje. Prepoznavanje
istaknute tvrdnje svodi se na detekciju odnosa između komentara i tvrdnje. 
Predložen je postupak za prepoznavanje istaknutih tvrdnji 
temeljen na nadziranom strojnom učenju. 

Naposljetku, treći zadatak u sklopu predistraživanja jest 
pronalazak implicitnih informacija u komentarima internetskih rasprava. 
Implicitne informacije između istaknute tvrdnje i komentara koji podupire 
tu istaknutu tvrdnju definirane su putem niza tvrdnji koje upotpunjavaju 
lanac logičkog zaključivanja. Predložene su metode za prepoznavanje istaknutih 
tvrdnji u komentarima uz korištenje implicitnih tvrdnji. 
Pokazano je kako korištenje implicitnih tvrdnji nedvojbeno pospješuje 
rješavanje zadatka prepoznavnaja istaknutih tvrdnji. 
Iz predistraživanja zaključeno je kako je pronalazak implicitnih informacija
bitan za kvalitetnu dubinsku analizu analizu argumentacije. 

Temeljem zaključaka iz predistraživanja, predložen je 
strukturirani pristup pronalaska istaknutih tvrdnji kroz 
radni okvir za strukturiranje tvrdnji. 
U takvome se pristupu definira formalna struktura tvrdnji kako bi se
umanjio problem različitih lingvističkih realizacija u tekstu i omogućilo
izvođenje implicitih tvrdnje logičkim zaključivanjem. 
Strukturirani pristup konceptualno je proveden u tri dijela. 
Prvo je predložena metoda za predviđanje tvrdnji iz komentara, 
zatim su tvrdnje modelirane i strukturirane pomoću računalnih ontologija,
te je, u konačnici, predložen niz metoda zasnovan na strukturiranom 
predviđanju za strukturiranje tvrdnji iz teksta. 

Problem previđanja tvrdnji iz komentara definiran je sukladno srodnim
problemima označavanja sekvenci, kao što je problem ekstrakcije imenovanih
entiteta.  Matematički je definiran problem predviđanja tvrdnji na dva načina.  Za oba načina
predloženo je više modela, među njima i model zasnovan na kombinaciji dubokog
učenja i strukturiranog previđanja. Predloženi modeli eksperimentano su vrednovani 
te je zaključeno kako metode strukturiranog previđanja pomažu
prilikom previđanja tvrdnji. 

S ciljem ublažavanja problema različitih lingvističkih realizacija 
i implicitnosti teksta, 
izdvojene tvrdnje strukturirane su pomoću računalnih ontologija. 
Kroz dvije razine računalnim ontologijama opisuju se tematski specifični koncepti te 
generički obrasci tvrdnji. Kako bi se strukturirale tvrdnje, prvo je 
potrebno definirati tematski specifične koncepte 
za svaku pojedinu temu rasprave, zatim je moguće kombinirati 
definirane koncepte s obrascima kako bi se tvrdnje strukturirale. 

Kako bi se dovršio zadnji korak strukturiranog radnog okvira za dubinsku
argumentativnu analizu predložene su metode za strukturiranje tvrdnji,
zasnovane na nadziranom strojnom učenju. Problem strukturiranja
tvrdnji pokazao se kao vrlo težak problem, uglavnom zbog velikog
broja mogućih rješenja. Eksperimentalnim vrednovanjem najboljim
metodama pokazala se metoda ulančanih klasifikatora,
zasnovanih na strukturiranom previđanju. 

Prednost strukturiranih tvrdnji za dubinsku analizu argumentacije 
demonstrirana je u vidu dohvaćanja implicitnih tvrdnji logičkim
zaključivanjem i grupiranjem sudionika temeljem zajedničkih
tvrdnji u raspravi. Time je pokazan objašnjiv i strukturiran
način pronalaska implicitnih tvrdnji.  

\vspace{1cm}
\textbf{Ključne riječi}:  
dubinska analiza argumentacije, obrada prirodnog jezika,
formalno predstavljanje znanja, 
strojno učenje, strukturirano predviđanje
