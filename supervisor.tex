\thispagestyle{empty}

\section*{About the Supervisor}


Jan Šnajder has received his BSc, MSc, and PhD degrees in Computer Science from
the University of Zagreb, Faculty of Electrical Engineering and Computing
(FER), Zagreb, Croatia, in 2002, 2006, and 2010, respectively. From September
2002 he was working as a research assistant, from 2011 as Assistant Professor,
and from 2016 as Associate Professor at the Department of Electronics,
Microelectronics, Computer and Intelligent Systems at FER. He was a
visiting researcher at the Institute for Computational Linguistics at the
University of Heidelberg, the Institute for Natural Language Processing at the
University of Stuttgart, the National Instituteof Information and
Communications Technology in Kyoto, and the University of Melbourne. He
participated in a number of research and industry projects in the field of
natural language processing and machine learning. He is the principal
investigator on a HRZZ installation grant project and a HAMAG-BICRO
proof-of-concept project, and a researcher on a UKF project. He has (co-)
authored more than 100 papers in journals and conferences in natural
language processing and information retrieval, and has been reviewing for major
journals and conferences in the field. He is the lecturer in charge for six
courses at FER and has supervised and co-supervised more than 100 BA and MA
theses. He is a member of IEEE, ACM, ACL, the secretary of the Croatian
Language Technologies Society, the co-founder and secretary of the Special
Interest Group for Slavic NLP of the Association for Computational Linguistics
(ACL SIGSLAV). He is a member of the Centre of Research Excellence for Data
Science and Advanced Cooperative Systems and the associate editor of the
Journal of Computing and Information Technology. He has been awarded the Silver
Plaque ``Josip Lončar'' in 2010, the Croatian Science Foundation fellowship in
2012, the fellowship of the Japanese Society for the Promotionof Science in
2014, and the Endeavour Fellowship of the Australian Government in 2015.


\section*{O mentoru}


Jan Šnajder diplomirao je,  magistrirao i doktorirao u polju računarstva na
Sveučilištu u Zagrebu Fakultetu elektrotehnike i računarstva (FER), 2002.,
2006. odnosno 2010. godine. Od 2002. godine radio je kao znanstveni novak, od
2011. godine kao docent, a od 2016. godine kao izvanredni profesor na Zavodu za
elektroniku, mikroelektroniku, računalne i inteligentne sustave FER-a.
Usavršavao se na Institutu za računalnu lingvistiku Sveučilišta u
Heidelbergu, Institutu za obradu prirodnog jezika Sveučilišta u Stuttgartu,
Nacionalnome institutu za informacijske i komunikacijske tehnologije u Kyotu
te Sveučilištu u Melbourneu. Sudjelovao je na nizu znanstvenih i stručnih
projekata iz područja obrade prirodnog jezika i strojnog učenja. Voditelj je
uspostavnog projekta HRZZ-a i projekta provjere koncepta HAMAG-BICRO-a te
je istraživač na projektu UKF-a. Autor je ili suautor više od 100 znanstvenih
radova u časopisima i zbornicima međunarodnih konferencija u području obrade
prirodnog jezika i pretraživanja informacija te je bio recenzentom za veći
broj časopisa i konferencija iz tog područja. Nositelj je šest predmeta na
FER-u te je bio mentorom ili sumentorom studentima na više od 100
preddiplomskih i diplomskih radova. Član je stručnih udruga IEEE, ACM, ACL,
tajnik Hrvatskoga društva za jezične tehnologije te suosnivač i tajnik posebne
interesne skupine za obradu prirodnog jezika za slavenske jezike pri udruzi za
računalnu lingvistiku (ACL SIGSLAV). Član je Znanstvenog centra izvrsnosti za
znanost o podacima i kooperativne sustave te je pridruženi urednik časopisa
Journal of Computing and Information Technology (CIT). Dobitnik je
Srebrne plakete ``Josip Lončar'' 2010. godine, stipendije Hrvatske zaklade za
znanost 2012. godine, stipendije Japanskog društva za promicanje znanosti
2014. godine te stipendije australske vlade Endeavour 2015. godine.
