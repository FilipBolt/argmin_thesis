\chapter{Conclusion}
\label{chap:conclusion}

% what was done -- summary
In the scope of doctoral thesis, we have researched multiple argumentation
mining problems in the domain of online discussions.  Argumentation mining
problems can be categorized according to where they lie in the argumentation
mining pipeline. The argumentation mining pipeline starts with text and
produces an argumentation structure. 
We worked on claim extraction, claim
structuring, and deriving implicit claims as part of our implementation of the
argumentation mining pipeline. To solve those problems, we make a distinction
between two approaches: an unstructured and a structured approach. In the
unstructured approach (part~\ref{part:unstruc}), we employ commonly-used machine
learning algorithms to extract natural language claims. On the other hand, in
the structured approach (part~\ref{part:struc}), we employ structured prediction
machine learning approaches. 
We opt for structured argumentation mining based on three pre-research studies
(described in chapters~\ref{chap:argclu}, \ref{chap:argrec}, and
\ref{chap:deriving_implicit}) which showed the limits of computational
representations of unstructured text. 

This work contains the following original scientific contribution. 
% Formalized claims
% (in comparison to natural language claims) contain a crisp logical
% representation of claims, which allows us to make inference to derive implied
% claims (chapter~\ref{chap:analysis}).

% reflection on contribution
This research proposes claim detection, structuring and analysis from online
discussions based on machine learning, natural language processing and
ontological modelling. Expected research contributions are: 1. Modeling an
online discussion using an two-level ontology, where the first level contains
domain knowledge and the second models claim patterns, aiming to fully
structure online discussions; 2. Supervised machine learning method for claim
detection and claim structuring; 3. Framework and support for online discussion
analysis involving claim detection, structuring, and analysis based on a
comparison of claims from all discussion participants


% future work, what more can be done in this area

% why is deriving implicit claims important
In general, we feel that deriving implicit claims is a very important, and
understated problem in not only argumentation mining, but also in computational
lingustics. To derive textual entailment, sentiment, In a recent news article,
the difficulty of acquiring implicit knowledge is highlighted
\citep{gradientpub}.  The news article criticizes the state-of-the-art language
model BERT \citep{devlin2018bert} to unravel that the state-of-the-art models
do not actually model implicit knowledge and human-level language
understanding, but exploit the statistical token-level cues often present in
datasets. We hope this research will inspire further research in argumentation
mining, in particular through exploring implicit claims and implicit knowledge. 
