\chapter{Conclusion}
\label{chap:conclusion}

% what was done -- summary
In the scope of the doctoral thesis, we have researched multiple argumentation
mining problems in online discussions.  Argumentation mining
problems can be categorized according to where they lie in the argumentation
mining pipeline. The argumentation mining pipeline starts with text and
produces an argumentative structure. 
We worked on claim extraction, claim
structuring, and deriving implicit claims as part of our implementation of the
argumentation mining pipeline. To solve those problems, we make a distinction
between two approaches: an unstructured and a structured approach. In the
unstructured approach (part~\ref{part:unstruc}), we employ commonly-used machine
learning algorithms to extract natural language claims. On the other hand, in
the structured approach (part~\ref{part:struc}), we used structured prediction
machine learning to obtain logical claim representations. 
We opt for structured argumentation mining based on three pre-research studies
(described in chapters~\ref{chap:argclu}, \ref{chap:argrec}, and
\ref{chap:deriving_implicit}) which showed the limits of computational
representations of unstructured text.  
The original research contribution of this thesis, based on 
structured argumentation mining, can be divided into three parts. 

The first part of the research contribution, described in
chapter~\ref{chap:formalization}, involves modeling an online discussion using a
argument framework to formalize claims.  
Formalized claims (in comparison to natural language claims) contain a crisp
logical representation of claims and allow us to infer additional knowledge. 
The framework consists of a two-level
ontology and verification steps to ensure the ontology quality of fit.  The
two-level ontology is composed of the upper and domain ontology.  The upper
ontology models general argumentative claim patterns, whereas the domain
ontology contains domain knowledge. To verify and validate the ontology
second-order description logic and textual entailment are used, respectively. 

The second part of the research contribution involves automatically building
formalized claims from text, which defines our argumentation mining pipeline.
We approach this problem in two steps.  In the first step, we build supervised
machine learning models to solve the claim segmentation problem in
chapter~\ref{chap:claim_segmentation}.  For the dataset, we expand the existing
dataset of online discussion comments \citep{hasan2014you} to involve claims
and claim formalizations.  In the second step, we take extracted claims from
claim segmentation and use supervised machine learning models to build claim
formalizations (described in chapter~\ref{chap:claim_structuring}). To
structure claims, we explore structured prediction approaches to take advantage
of the dependencies of claim formalization components.  To build formalized
claims directly from text, we also propose a supervised joint model which is
trained to do claim segmentation and claim structuring in one step. 

The third, and final, part of the research contribution we build a framework
and support for online discussion analysis involving claim detection,
structuring, and analysis based on a comparison of claims from all discussion
participants. In chapter~\ref{chap:analysis} we build a domain ontology based
on the ``\emph{Marijuana}'' topic, integrate it with the upper ontology,
annotate claim individuals, and validate the end result. We train and
evaluate claim structuring and claim segmentation models (in
chapters~\ref{chap:claim_segmentation} and~\ref{chap:claim_structuring}) which
produce formalized claims. Finally, we showcase the advantages of having claims
formalized by deriving implicit claims.  We derive implicit claims by comparing
made claims across groups of speakers. 

As part of the research contribution we propose a workflow to analyze
the argumentative claims of 
and model a single discussion topic, the ``\emph{Marijuana}'' topic. 
However, the workflow 

\todo{dovrsi zakljucak}

However, 
the workflow is described in terms such that applying the workflow
to a new topic has clearly defined st

The research contribution defines a workflow to model 
a single discussion topic. We apply this workflow to model the 
``\emph{Marijuana}'' topic, but propose a workflow general steps to apply
them to any other discussion topic. Having general steps 

making the research contribution
applicable to any argumentative discussion.

% future work, what more can be done in this area

% why is deriving implicit claims important
In general, we feel that deriving implicit claims is a very important, and
understated problem in not only argumentation mining, but also in computational
lingustics. To derive textual entailment, sentiment, In a recent news article,
the difficulty of acquiring implicit knowledge is highlighted
\citep{gradientpub}.  The news article criticizes the state-of-the-art language
model BERT \citep{devlin2018bert} to unravel that the state-of-the-art models
do not actually model implicit knowledge and human-level language
understanding, but exploit the statistical token-level cues often present in
datasets. We hope this research will inspire further research in argumentation
mining, in particular through exploring implicit claims and implicit knowledge. 


% what methods need improvement
\todo{break down into sentences}
- claim segmentation could be improved with the heuristic, 
- multi-label approach needs further exploration 
- claim structuring may work better with constrained programming, 
as everything can be expressed as equality and inequalities (Lagrange)
- ontology formalizations are richer than microstructures
