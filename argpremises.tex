\chapter{Deriving Implicit Claims}
\label{chap:deriving_implicit}

In the previous chapter, we built a model for identifying 
prominent claims in a comment. While doing so, we noticed that 
some prominent claims were either implicitly or explicitly mentioned
in user comments. This was modelled in the task of prominent claim
identification, but now we wish to explore the implicit knowledge required
to state that a prominent claim is indeed supported or attacked by a comment.
To this end, we shall investigate how and when are two claims
(in the setting of online discussions) considered equivalent. 

To make two claims semantically equivalent, sometimes \textbf{implicit knowledge} is
required to bridge the gap between the two claims. 
Having this implicit knowledge might improve performance in the 
prominent claim identification problem. 
Many factors contribute to the gap between two claims: linguistic variation,
implied commonsense knowledge, or implicit claims from beliefs and value
judgements of the person making the claim. In table~\ref{tab:premise_example} we
give an example from the dataset of~\citet{hasan2014you}. 
In the example, the user comment: 
\emph{Now it is not taxed, and those who sell it are usually criminals of some sort.}
is matched to \pro{support} the prominent claim stating
\emph{If something is not taxed, criminals sell it}.
Without the additional implicit claims, the user comment does not support the prominent 
claim because of an inference gap. 

% TODO add section overview

\begin{table}
{\normalsize
\begin{tabular}{|@{\ }r@{\ \  }p{0.8\columnwidth}|}
\hline
\textbf{User comment:} & \emph{Now it is not taxed, and those who sell it are
	usually criminals of some sort.}\\
\textbf{Prominent claim:} & \emph{Legalized marijuana can be controlled and
	regulated by the government.}\\
\textbf{Implicit Claim 1:} & \emph{If something is not taxed, criminals sell
	it.}\\
\textbf{Implicit Claim 2:} & \emph{Criminals should be stopped from selling
	things.}\\
\textbf{Implicit Claim 3:} & \emph{Things that are taxed are controlled and
	regulated by the government.}\\
\hline
\end{tabular}}
\caption{User claim, the matching prominent claim, and the implicit claims filling the gap.}
\label{tab:premise_example}
\end{table}

\section{Data}
\label{sec:argpremise_dataset}

\begin{table}
\begin{center}
{\small
\begin{tabular}{lcc}
\toprule
Topic & \#\,claim pairs  & \#\,main claims \\
\midrule
``\emph{Marijuana}'' (MA)	   & 125                     &  10                    \\
``\emph{Gay Rights}'' (GR)	   & 125                     &  9                     \\
``\emph{Abortion}'' (AB)	   & 125                     &  12                    \\
``\emph{Obama}'' (OB)	       & 125                     &  16 \\
\bottomrule
\end{tabular}}
\caption{Dataset summary. }
\label{tab:argpremise_topic_distribution}
\end{center}
\end{table}


\begin{table}
{\small
\begin{tabular}{@{}p{0.55\columnwidth}p{0.4\columnwidth}@{}}
\toprule
Claim pair & Annotation            \\
\midrule
 \textbf{User claim:} \emph{Obama supports the Bush tax cuts. He did not try to
	end them in any way.} & \textbf{P1:}  \emph{Obama continued with the
	Bush tax cuts.}      \\
 \textbf{Prominent claim:} \emph{Obama destroyed our economy.}  & \textbf{P2:}
	\emph{The Bush tax cuts destroyed our economy.}   \\
\midrule
 \textbf{User claim:} \emph{What if the child is born and there is so many
	difficulties that the child will not be able to succeed in life?}  &
	\multirow{2}{*}{Non-matching}   \\
 \textbf{Prominent claim:} \emph{A fetus is not a human yet, so it's okay to abort.}
	&   \\
\midrule
 \textbf{User claim:} \emph{Technically speaking, a fetus is not a human yet.} & \multirow{2}{*}{Directly linked}      \\
 \textbf{Prominent claim:} \emph{A fetus is not a human yet, so it's okay to abort.}     & \\
\bottomrule
\end{tabular}}
\caption{Examples of annotated claim pairs.}
\label{tab:argpremises_example}
\end{table}

As a starting point, we use the dataset created by
\citet{hasan2014you}. 
The dataset contains posts from a two-side online unmoderated discussion platform 
on four topics: ``\emph{Marijuana}'' (MA), ``\emph{Gay Rights}'' (GR), 
``\emph{Abortion}'' (AB), and ``\emph{Obama}'' (OB).
Each post from the discussion is assigned a stance label (\pro{pro} or \con{con}) 
provided by the author of the post (the author of the post has to ``pick'' a side 
with respect to the topic). 
Each post is split into sentences, with each sentence manually
labelled with a single claim from a predefined set of prominent
claims. 
\citet{hasan2014you} report significantly high agreement on the labelings
(from 0.61 to 0.67 $\kappa$ score, depending on the topic). 
Our annotation extends this dataset. 
We formulate a ``fill-the-gap'' task. 
Given a pair of previously matched claims (user comment and 
prominent claim), we ask annotators to provide implicit claims that bridge the gap 
between the two claims. 
No further instructions were given to the annotators (this is contrary to all other
annotation tasks where guidelines were usually very detailed, examples can be seen in appendix
~\ref{chap:annotation_guidelines}). 
We hoped they would resort to common-sense reasoning and effectively 
reconstruct the deductive steps needed to entail the prominent claim from the user
claim.
The annotators where free to abstain from filling the gap, they
annotated such examples as \emph{Non-matching}. 
On the other hand, if they thought no implicit claims are required
to bridge the gap they annotated the pair of claims as \emph{Directly linked}. 
We hire three annotators (we denote them as A1, A2, and A3) to annotate each pair of claims. 
The order of claims is randomized for each annotator. 
We annotate 125 claim pairs for each topic, yielding a total of 500 gap-filling
\textbf{implicit claim sets}. 
Table~\ref{tab:argpremise_topic_distribution} shows the prominent claim distribution per topic. 
An excerpt from the dataset is in table~\ref{tab:argpremises_example}. 
We make the dataset freely available.\footnote{Available under the CC BY-SA-NC license from
\url{http://takelab.fer.hr/argpremises}}

\section{Implicit Claim Analysis}
\label{sec:argpremise_analysis}

\begin{table}[t]
{\small
\begin{center}
\begin{tabular}{lrrrr}
\toprule
& A1 & A2 & A3 & Avg.\\
\midrule
Avg.~\#\,implicit claims  & 3.6  & 2.6   & 2.0   &  \phantom{0}2.7 $\pm$ 0.7  \\
Avg.~\#\,words     & 26.7 & 23.7  & 18.6  &  23.0 $\pm$ 3.4      \\
Non-matching (\%)     & 1.2  & 3.6   & 14.5  &  \phantom{0}6.4 $\pm$ 5.8  \\
\bottomrule
\end{tabular}
\caption{Gap-filling parameters for the three annotators.}
\label{tab:var-annotators}
\end{center}}
\end{table}

The aim of the first study is to analyze how people fill 
the gap between the user's claim and the corresponding prominent claim. 
We pose three research questions for the first study: 
\begin{enumerate}[label=\arabic*)]
\item \textbf{gap variability}: to what extent of variability do different people fill the gap,
\item \textbf{gap characterization}: what types of implicit claims are used to fill the gap, and
\item \textbf{semantic similarity of claims}: how the gap relates to the more general notion of 
textual similarity between claims. 
\end{enumerate}
The dataset is adopted from \citet{hasan2014you} which brings three issues. 
First, the prominent claim, which has already annotated, need not be
the correct one. 
We remedy this by asking our annotators to abstain from adding implicit claim if 
they believe so, but we rarely expect this to be the case. 
The second issue is the granularity of prominent claims. 
As noted in chapter~\ref{chap:argclu}, the level of claim granularity is 
arbitrary to a certain extent. 
We speculate that the smaller the predefined set of prominent claims the 
bigger will the average gap between user claims and prominent claims be. 
Finally, as each gap was not filled by the same person who annotated the original 
dataset, as the original author might have chosen a different set of implicit claims than those
ascribed to by our annotators. 
Considering the above, we cannot analyze the \emph{genuine} implicit claims
of the claim's author. 
However, we hope our annotators fill the gap with a set of \emph{sensible} implicit 
claims. 
Depending on how appropriate the prominent claim was, this gap will be larger or smaller. 

\subsection{Variability in Gap Filling} 

\begin{table}[t!]
{\small
\begin{tabular}{@{\ }r@{\ \  }p{0.72\columnwidth}}
\toprule
\textbf{User claim:} & \emph{It would be loads of empathy and joy for about 6
hours, then irrational, stimulant-induced paranoia. If we can expect the former
to bring about peace on Earth, the latter would surely bring about WWIII.}\\
\textbf{Prominent claim:} & \emph{Legalization of marijuana causes crime.}\\
\midrule
\textbf{A1 implicit claim 1:} & \emph{Marijuana is a stimulant.}\\
\textbf{A1 implicit claim 2:} & \emph{The use of marijuana induces paranoia.}\\
\textbf{A1 implicit claim 3:} & \emph{Paranoia causes war.}\\
\textbf{A1 implicit claim 4:} & \emph{War causes aggression.}\\
\textbf{A1 implicit claim 5:} & \emph{Aggression is a crime.}\\
\textbf{A1 implicit claim 6:} & \emph{"WWIII" stands for the Third World War.}\\
\midrule
\textbf{A3 implicit claim 1:} & \emph{Marijuana leads to irrational paranoia which can lead to committing a crime.} \\
\bottomrule
\end{tabular}}
\caption{User claim, the matching prominent claim, and the implicit claim(s)
filling the gap provided by two different annotators.}
\label{tab:extreme_premisenumber}
\end{table}

To characterize the variability, we calculate the following quantitative parameters:
\begin{enumerate*}[label=(\arabic*)]
\item the average number of implicit claims, 
\item the average number of words in implicit claims, and
\item the proportion of non-matched claim pairs.
\end{enumerate*}
Table~\ref{tab:var-annotators} shows substantial variability in these parameters for the three
annotators. 
Average number of implicit claims per gap is 2.7 and the average number of words per gap is 23, 
yielding the average length of about 9 words per implicit claim. 
We also computed the word overlap between the three annotators: 8.51, 7.67, and 5.93 
for annotator pairs A1--A2, A1--A3, and A2--A3, respectively. 
This indicates that, on average, implicit claim sets overlap in just 32\% of words. 
The annotators A1 and A2 have a higher word overlap and use more words to fill the gap. 
Also, A1 and A2 managed to fill the gap for more cases than A3 who 
often desisted from filling the gap. 
An example where A1 used considerably more implicit claims than A3 is shown in 
table~\ref{tab:extreme_premisenumber}. 
In table~\ref{tab:var-topics} gap-filling metrics are shown across topics. 
Here the picture is more balanced. 
The least number of implicit claims and least number of words are used for the
``Abortion'' topic. The ``Gay Rights'' topic contains the most (about 7\%)
claim pairs for which the annotators desisted from filling the gap. 

\subsection{Gap Characterization}

\begin{table}
{
\begin{center}
\setlength{\tabcolsep}{4.2pt}
\begin{tabular}{@{}lrrrrr@{}}
\toprule
&\multicolumn{4}{c}{Topic}\\
\cmidrule(lr){2-5}
& ``\emph{Marijuana}'' & ``\emph{Gay Rights}'' & ``\emph{Abortion}'' & ``\emph{Obama}'' & Avg. \\
\midrule
Avg.~\#\,implicit claims  & 2.8  & 2.8   & 2.5   &  2.8  &  \phantom{0}2.7 $\pm$ 0.1 \\
Avg.~\#\,words     & 23.6  & 24.9   & 19.1   &  23.4  & 22.8 $\pm$ 2.2\\
Non-matching (\%)     & 5.9  & 6.8   & 4.6   &  4.3  &  \phantom{0}5.4 $\pm$ 1.0\\
\bottomrule
\end{tabular}
\caption{Gap-filling parameters for the four topics.}
\label{tab:var-topics}
\end{center}}
\end{table}

After a preliminary inquiry into the nature of the gap, 
now we wish to characterize the implicit claims that fill the gap. 
We do not look at the relationships between the implicit claims (argumentative
structure). 
We categorize implicit claims along three dimensions: 
\begin{itemize}
\item type (fact, value, policy), 
\item complexity (atomic, implication, or complex), and
\item acceptance (universal or claim-specific).
\end{itemize}
The intuition behind acceptance is that some implicit claims convey general truths or widely
accepted beliefs, while others are specific to the claim being made, and embraced
only by the supporters of the claim in question. 
We manually classified 50 implicit claims from the ``Marijuana'' topic into the
above categories and averaged the proportions. 
Our annotation agreement Cohen's kappa ($\kappa$) \citep{cohen1960coefficient} is 0.42, 
0.62, and 0.53 for the implicit claim type, complexity, and acceptance, respectively.
Factual implicit claims account for the large majority of cases (85\%), value implicit claims
for 9\%, and policy implicit claims for 6\%.
Most of gap-filling implicit claims are atomic (77\%), while implication and other complex
types constitute 16\% and 7\% of cases, respectively.
In terms of acceptance, implicit claims are well-balanced: universal and claim-specific 
implicit claims account for 62\% and 38\% of cases, respectively. 
We suspect this kind of analysis might be relevant for determining the overall 
strength of an argument (a problem dealt with in \citep{park2014identifying}). 
Of course, this study can be extended to the entire dataset and involve
a larger (unbiased) set of annotators. 

\subsection{Claim Semantic Similarity}

In the previous chapter, we addressed prominent claim identification as a semantic
textual similarity task. 
Therefore, we attempt to use semantic textual similarity to characterize the gap
between two claims. 
We hypothesize that the textual similarity between two claims will be
negatively affected by the size of the gap. 
Thus, even though the claims are matching, if the gap is too big, similarity will be not 
high enough to indicate the match. 
To verify this, we compare the semantic similarity score between each pair of 
claims against its gap size, characterized by the number of implicit claims
required to fill the gap, 
averaged across three annotators. 
To obtain a reliable estimate of semantic similarity, we rely on human-annotated 
similarity judgements. 
We setup a crowdsourcing task where we asked workers to judge similarity
between 846 claim pairs for the ``\emph{Marijuana}'' topic. 
The task was formulated as a question ``\emph{Are two claims
talking about the same thing?}'', and judgements were made on a scale
from 1 (``not similar'') to 6 (``very similar''). 
Annotation guidelines are detailed in the appendix
section~\ref{sec:argpremises_annotation}. 
Each pair of claims received five judgements, which we averaged to obtain the
gold-similarity score. 
The average standard deviation is 1.2, indicating good agreement. 

The Paerson correlation coefficient  between the similarity score and the
number of implicit claims filling the gap for annotators A1, A2, and A3 is
$-0.30$, $-0.30$, and $-0.14$, respectively.  The correlation coefficient between the
similarity score and the number of implicit claims averaged across all annotators is
$-0.22$ ($p < 0.0001$).  We conclude there is a statistically significant,
albeit weak, negative relationship between semantic similarity and gap size. 

Thus far, we've demonstrated the degree of similarity between implicit 
claims and how it is negatively correlated with the gap size. 
This suggests that the similarity scores could be increased by reducing the size 
of the gap.
Based on that conclusion, we expect to reduce the size of the gap if we start
including implicit claims in the computations. 
We make a preliminary study on the use of implicit claims in prominent claim
identification.
This study is motivated to identify prominent claims in online discussions in
general.
Given the users claim , the task is to find the prominent claim from a set of 
prominent claims which matches the user's claim the best. 
We pose \emph{three research questions}: 
\begin{enumerate*}[label=(\arabic*)]
\item whether and how the use of implicit claims influences prominent claim identification, 
\item how well do implicit claims generalize, and 
\item could the implicit claims be automatically retrieved.
\end{enumerate*}

\section{Prominent Claim Identification with Implicit Claims}
\label{sec:argpremise_identification}

The prominent claim identification task can be approached in a supervised or
unsupervised manner (supervised and unsupervised approaches are described in 
sections and~\ref{sec:unstruc_machine_learning} and
\ref{sec:unsupervised_machine_learning} respectively)
We focus on the unsupervised approach. 
We use the semantic similarity between claims and implicit claims, as
unsupervised prominent claim identification provides a more straightforward and
explicit way of incorporating implicit claims. 
Further, the unsupervised approach better corresponds to the very idea of
argumentation, 
where claims and implicit claims are compared to each other and combined to 
derive additional claims. 

We use the dataset described in section~\ref{sec:argpremise_dataset} 
consisting of 125 claim pairs for four topics. 
We use gap filling implicit claims from annotator A1, who, on average, has the
highest number of implicit claims. 
We refer to this dataset as the \emph{development set}.
In addition, we sample an additional \emph{test set} consisting of 125 pairs for each topic
from the dataset of \citet{hasan2014you}. 
For claim pairs from this set we have no
implicit claims (in this \emph{test set}). 

We adopt the distributional semantics approach to compute semantic textual similarity. 
We rely on distributed representations \citep{mikolov2013distributed} to represent text. 
More specifically, we represent text of claims and premises by summing up distributional
vectors of individual words and measure similarity as cosine similarity. 
We also considered using textual entailment,
but due to poor results with \textit{Excitement Open Platform}
\citep{pado2015design}, we omit its results from this work. 

We employ two baselines.  First, an unsupervised baseline, which simply
computes the similarity between the users claim and the prominent claim vectors
without using implicit claims. 
Each user claim is matched against the most similar prominent claim. 
The other is a supervised baseline which uses support vector machine (SVM)
(described in section~\ref{sec:svm}) with an RBF kernel, trained 
on user comments to predict the label corresponding to the 
prominent claim. 
We train and evaluate using a $5 \times 3$ nested cross validation , separately for each topic. 
The hyperparameters $C$ and $\gamma$ are optimized using grid-search
(model selection is described in section~\ref{sec:selection}). 
We use the LibSVM implementation \citep{chang2011libsvm}. 

\begin{table}
\begin{center}
{\small
{\def\arraystretch{1.2}\tabcolsep=2pt
\begin{tabular}{@{}lp{13cm}@{}}
\toprule
Type & Text content  \\
\midrule
$U_i$      & \emph{Marijuana has so many benefits for sick people.} \\
$M_j$    & \emph{Marijuana is used as a medicine for its positive effects.}   \\
$P_{ij}$     & \emph{Marijuana helps sick people. Sick people use marijuana.} \\
\midrule
$U_i + P_{ij}$   & \emph{Marijuana has so many benefits for sick people.
Marijuana helps sick people. Sick people use marijuana.} \\ 
$M_j + P_{ij}$   & \emph{Marijuana is used as a medicine for its positive
effects. Marijuana helps sick people. Sick people use marijuana.}\\
\bottomrule
\end{tabular}}}
	\caption{Combination of implicit claim sets ($P_{ij}$) with  prominent
	($M_j$) and user ($U_i$) claims}
\label{tab:argpremise_concatenation}
\end{center}
\end{table}

To obtain a single combined representation of a implicit claim set, we simply
concatenate the implicit claims together before computing the distributional vector
representation. 
We do the same when combining the implicit claims with either of the (prominent or user) claim.
This is exemplified by table~\ref{tab:argpremise_concatenation}. 
We denote the user claim, prominent claim and gap filling implicit claim set between 
user claim and prominent claim as
$U_i$, $M_j$, and $P_{ij}$ respectively. 

%\subsection{Prominent Claim Identification with Implicit Claims}

\begin{table}
\begin{center}
{\small
\setlength{\tabcolsep}{5.9pt}
\begin{tabular}{lrrrrrr}
\toprule
&\multicolumn{4}{c}{Topic}\\
\cmidrule(lr){2-5}
Model & MA & GR  & AB & OB & Avg. \\
\midrule
% napraviti alignment tako da je vs. dio uvijek na istom mjestu
$U_i \leftrightarrow M_j$      & 7.39          & 12.52        & 24.59        & 10.87        & 13.84 \\
$U_i \leftrightarrow M_j$\ (S)  & 35.26         & 27.81        & 33.30        & 20.92        & 29.32 \\
$U_i + P_{ij} \leftrightarrow M_j$   & 22.73         & {\bf 46.03}  & 47.22        & 21.41        & 34.35 \\
$U_i \leftrightarrow M_j + P_{ij} $ & {\bf 48.05}   & 28.23        & {\bf 49.34}  & {\bf 64.11}  & {\bf 47.43} \\
\bottomrule
\end{tabular}}
\caption{Performance of claim matching baselines and oracle performance of the
claim matching models utilizing implicit claims from annotator A1
(macro-averaged F1-score).}
\label{tab:argpremise_matching}
\end{center}
\end{table}

First, we wish to know whether implicit claim sets can help in 
prominent claim identification at all. We use gold-annotated 
implicit claim sets and combine these with either the prominent
or the user claim. 
The idea is that by combining the implicit claims with a claim, we encode 
the information conveyed by the implicit claims into the claim, 
hopefully making claims more similar at the textual level. 
We consider four models: \begin{enumerate}[label=\arabic*)]
\item the unsupervised baseline (``$U_i \leftrightarrow M_j$''),
\item the supervised baseline (``$U_i \leftrightarrow M_j (S)$''), 
\item the model in which the implicit claims are combined with the user claim,
	(``$U_i+P_{ij}\leftrightarrow~M_j$''), and 
\item the model in which the implicit claims are combined with the prominent claim  \\
	(``$U_i~\leftrightarrow~M_j+P_{ij}$''). 
\end{enumerate}
The latter two predict the prominent claim as the one that maximizes the 
similarity between two claims, after one of the claims is combined with implicit claims.
The $U_i + P_{ij} \leftrightarrow M_j$ model considers all pairs of the user claim $U_i$ 
and gold-annotated implicit claim sets $P_{i*}$ for that claim $i$. 
In contrast, the $U_i \leftrightarrow M_j + P_{ij}$ model considers all pairings
of the main claim $M_j$ and the gold-annotated implicit claim sets $P_{*j}$ for the prominent 
claim $j$. 
In effect this model tries to fill the gap using different implicit claim sets linked
to the given prominent claim. 
In this oracle setup, we always use gold-annotated implicit claim set for the
prominent claim. 

In table~\ref{tab:argpremise_matching} we show claim matching results in terms of
macro-averaged F1-score on the development set. 
Results suggest that using implicit premises helps in selecting the most similar prominent claim 
as the models with added implicit claims outperform the unsupervised baseline by
20.5 and 33.6 points F1-score. 
Furthermore, the model that combines implicit claims with the prominent claim considerably 
outperforms both baselines and the model that combines implicit claims with the user claim. 
An exception is the ``\emph{Gay Rights}'' topic, on which the latter model works better. 
Our analysis revealed this to be due to the presence of very general 
(i.e. lexically non-discriminative) implicit claims in some implicit claim sets 
(e.g. ``\emph{Straight people
have the right to marry}'') which makes corresponding prominent claims more similar to
user claims. 
Another interesting observation is very good performance on the ``\emph{Obama}'' topic. 
This is likely because only one of 16 prominent claims contains the word ``Obama''
making it more similar to user claims. 
However, after implicit claim sets get combined with all prominent claims this
difference diminishes and prominent claim identification performance improves. 
We obtained results from table~\ref{tab:argpremise_matching} using implicit claims compiled by
annotator A1. 
To see how model performance is affected by using different implicit claim
sets, we re-run the same experiment with best-performing $U_i \leftrightarrow
M_j + P_{ij}$ model, this time using implicit claim sets compiled by A2 and A3. 
Although we obtained a lower macro-averaged F1 score (33.97 for A2 and 32.91 for A3) the model 
still outperforms both baselines. 
On the other hand, this does suggest that performance can vary greatly with the quality of 
implicit claim sets. 

% TODO maybe move this to the discussion part 

The prominent claim identification problem resembles query matching in
information retrieval. 
One common way to address the lexical gap in information retrieval is to perform
query expansion \citep{voorhees1994query}. 
We hypothesize that human-compiled implicit claims are more useful for prominent claim
identification than standard
query expansion. 
To verify this, we replicate setups $U_i + P_{ij} \leftrightarrow M_j$ and
$U_i \leftrightarrow M_j + P_{ij}$, but instead of implicit claim sets, use
\begin{enumerate*}[label=(\arabic*)]
\item WordNet \citep{miller1995wordnet} synsets and
\item top $k$ distributionally most similar words (using distributed
word representations described in section~\ref{sec:embedding} for $k=\{1, 3, 5, 7, 9\}$) to
expand the user or prominent claim.
\end{enumerate*}
We obtained no improvement over baselines suggesting that lexical information 
in the implicit claim sets is indeed specific. 

\section{Implicit Claim Generalization}
\label{sec:argpremise_generalization}

\begin{table}[t]
\begin{center}
{\small
\setlength{\tabcolsep}{5.9pt}
\begin{tabular}{lrrrrrr}
\toprule
&\multicolumn{4}{c}{Topic}\\
\cmidrule(lr){2-5}
Model & ``Marijuana'' & ``Gay rights''  & ``Abortion'' & ``Obama'' & Avg. \\
\midrule
$U_k \leftrightarrow M_j$   & 9.60          & 19.68        & 27.70        & 12.39        & 17.35 \\
$U_k \leftrightarrow M_j$\ (S)   & 29.01         & {\bf 29.39}  & 21.09        & 18.22        & 24.43 \\
$U_k \leftrightarrow M_j + P_{ij}$  & {\bf 30.63}   & 23.00        & {\bf 32.72}  & {\bf 23.87}  & {\bf 27.55} \\
\bottomrule
\end{tabular}}
\caption{Performance of prominent claim identification baselines and the models utilizing the
implicit claims on the test set (macro-averaged F1-score).}
\label{tab:argpremise_generalization}
\end{center}
\end{table}

From a practical perspective, we are interested to what extent the implicit claims
generalize, whether it is possible to reuse the premises compiled for the
prominent claims, but different user claims. 
We choose the best performing model from the previous section ($U_i
\leftrightarrow M_j + P_{ij}$)
and apply this model and the baseline models on the \emph{test set}. 
This means that the model uses implicit claim sets $P_{ij}$ for pairs of claims
$U_i$ and $M_j$ from the training set, and hope is that the same implicit claim
sets will be useful for unseen user claims $U_k$. 
Results are shown in table~\ref{tab:argpremise_generalization}. 
Model again outperforms the baselines, except on the ``\emph{Gay rights}'' topic. 
The performance varies across topics: the average improvement over unsupervised
and supervised baselines is 10.2 and 3.12 points of F1-score, respectively. 
This result suggests that the implicit claim sets that fill the gap generalize to a certain
extent and thus can be reused for unseen user claims. 

\section{Implicit Claim Retrieval}
\label{sec:argpremise_retrieval}

\begin{algorithm}[t]
\begin{algorithmic}[1]
\State \textbf{input}: claim $C$, prominent claim set $M$
\State \textbf{output}: implicit claim set $P$
\State
\State $S_{avg} \gets \frac{1}{|M|} \sum_{m \in M} \mathtt{sim}(m, C)$
\For{$i \in \{1, 2, 3, 4, 5\}$}
  \State $P_{sim} \gets \mathtt{most\_similar}(C, i)$
  \State $C_{new} \gets C + P_{sim}$
  \State $S_{new} \gets \frac{1}{|M|} \sum_{m \in M} \mathtt{sim}(m, C_{new})$
  \State
  \If {$S_{new} > S_{avg}$}
    \State $i \gets i + 1$
    \State $P \gets P \cup P_{sim}$
  \Else
    \State \textbf{break}
  \EndIf
  \State
\EndFor
\end{algorithmic}
\caption{Implicit claim retrieval heuristic where \texttt{most\_similar} is a
	function that returns most similar implicit claims given a claim and
	desired number of most similar items and \texttt{sim} is a function that
	returns a similarity score given two texts as input}
\label{alg:premise_retrieval}
\end{algorithm}

In a realistic setting, we don't have access to implicit premises 
for each prominent claim, but we try to fetch or generate them automatically. 
This is a preliminary study where we investigate the feasibility of fetching 
premises automatically. 
Given the claim $C$ as input and the predefined prominent claim set $M$, the
goal is to find the implicit claim set $P$ that would improve prominent claim
identification for the claim $C$. 
To retrieve the implicit claim set $P$ and then perform prominent claim
identification, we use a simple greedy heuristic. 
Choose $i$ implicit claims most similar to the user claim, then combine them
with the user claim.
While the user claim is increasingly similar to the prominent claims (which
means it should be easier to match), we retrieve additional similar claims. 
Once the similarity to prominent claims stops increasing, we stop
adding fetched implicit claims to the implicit claim set. The procedure 
is outlined by algorithm~\ref{alg:premise_retrieval}.
We consider two setups: 
\begin{enumerate*}[label=(\arabic*)]
\item one in which the pool of implicit claims to retrieve
from comes from the topic in question (within-topic) and 
\item other in which implicit claims from all four topics are considered (cross-topic). 
\end{enumerate*}
Results are shown in table~\ref{tab:argpremise_retrieval}. 
We evaluate on both the development and test set, as well as within topic (WT)
and cross-topic (XT). 
Results show that our simple method for within-topic premise retrieval improves
prominent claim identification over the baseline for all topics except the ``\emph{Obama}'' topic. 
On the other hand, test set results suggest that the model does not generalize well, 
as it does not outperform the baseline. 

\begin{table}
\begin{center}
{\small
\setlength{\tabcolsep}{4.8pt}
\begin{tabular}{lrrrrrr}
\toprule
&\multicolumn{4}{c}{Topic}\\
\cmidrule(lr){2-5}
Model & ``Marijuana'' & ``Gay rights''  & ``Abortion'' & ``Obama'' & Avg. \\
\midrule
$U_i \leftrightarrow M_j$ & 7.39          & 12.52        & 24.59       & {\bf 10.87} & 13.84 \\
$U_i + P \leftrightarrow M_j$\ (WT)     & {\bf 8.95}    & {\bf 19.54}  & {\bf 29.32} & 7.30        & {\bf 16.28} \\
$U_i + P \leftrightarrow M_j$ \ (XT)   & 8.56          & 19.01        & 28.73       & 7.07        & 15.84 \\
\midrule
$U_k \leftrightarrow M_j$            & {\bf 9.60}   & {\bf 19.68}   & {\bf 27.70} & 12.39       & {\bf 17.35} \\
$U_k \leftrightarrow M_j$\ (XT)  & 5.69         &  17.75        & 15.38       & {\bf 12.43} & 12.82 \\
\bottomrule
\end{tabular}}
\caption{Performance of the claim matching model with implicit claim retrieval on the
dev.~set (upper part) and test set (lower part); macro-avg.~F1-score.}
\label{tab:argpremise_retrieval}
\end{center}
\end{table}

\section{Conclusion}
\label{sec:argpremise_conclusion}

We addressed the problem of prominent claim matching of user claims to prominent claims
using implicit claims.
The gap between user and prominent claims introduces a gap that is 
easily filled by humans, but difficult to bridge for natural language processing 
methods. 
First, we compiled a dataset of implicit claims between matched claims from 
online discussions.
We showed considerable variation in the way how human annotators fill the gap
with implicit claims and they use implicit claims of various types. 
Also, we showed that similarity between claims, as judged by humans, negatively
correlates with the size of the gap, 
expressed in the number of implicit claims needed to fill it. 
Second, we experimented with computational models for prominent claim identification 
which used implicit claims. 
We showed that using gap filling implicit claims effectively reduces similarity gap between claims 
and improves prominent claim identification. 
We showed that implicit claim sets generalize to certain extents, as prominent claim matching was
improved on unseen user claims. 
Finally, we made a preliminary study on how to retrieve gap-filling implicit
claim sets automatically.

There is a great deal of usefulness in implicit claims, helping prominent claim
identification is just one small aspect of it. But, as we have shown in this chapter,
retrieving implicit claims is difficult using simplistic approaches presented
so far. We feel comparing claims with some semantic similarity measure is not 
nearly enough to capture the nuances of the different relationships two claims can have.
Prominent claim identification, claim clustering and deriving implicit
claims have all thus far been employed at a claim level. To be able to 
determine relationship between two claims (entailment, paraphrase, implication)
we will now focus on exploring beyond the claim level: the \textbf{sub-claim level}. 
In the next part, we will deal with similar problems, such as deriving
implicit claims, but we will go with a fundamentally different, and to the best
of our knowledge, novel approach in argumentation mining. Automatically retrieving 
implicit claims will then be again revisited in 
chapter~\ref{chap:analysis}, but using a structured approach.
