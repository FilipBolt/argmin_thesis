\section{Implicit claim annotation}
\label{app:sec:argpremises_annotation}

\subsection*{Annotation guidelines}

%This produced the argpremises dataset similarities in the (\ref{item:argpremises}).

IMPORTANT: This is essentially a reading comprehension task. While we
appreciate your opinion on this topic, this task is about analyzing OTHER
PEOPLE'S OPINIONS, not expressing your own. This is not an online survey.

\noindent There is an online discussion on marijuana legalization. The topic is: ``SHOULD
MARIJUANA BE LEGALIZED?''. You have to be familiar with of the issue: You can
get basic information about the topic 
in the footnotes\footnote{\url{http://medicalmarijuana.procon.org/}}\footnote{
\burl{http://idebate.org/debatabase/debates/health/addiction/
house-believes-cannabis-should-be-legalised}
}
Your goal is to identify whether the two sentences offered are talking about
the same thing. Please rate the SIMILARITY LEVEL for each pair of sentences:
\begin{itemize}
\item 6: very similar
\item 5 or 4: somewhat similar
\item 3 or 2: somewhat dissimilar
\item 1: not similar
\end{itemize}

\noindent Example one:
\begin{table}[h!]
\begin{tabular}{|@{\ }r@{\ \  }p{0.72\columnwidth}|}
\hline
\textbf{Sentence 1:} & \emph{Marijuana does not cause any damage to our bodies}\\
\textbf{Sentence 2:} & \emph{Consuming pot never hurt anyone}\\
\textbf{Similarity level} & 6 \\
\hline
\end{tabular}
\end{table}
\pagebreak
%\caption{User claim, the matching main claim, and the implicit premises filling the gap.}

\noindent Example two:
\begin{table}[h!]
\begin{tabular}{|@{\ }r@{\ \  }p{0.72\columnwidth}|}
\hline
\textbf{Sentence 1:} & \emph{If legalized, people will use marijuana and other drugs more}\\
\textbf{Sentence 2:} & \emph{Marijuana can be used as medicine because it showed positive effects}\\
\textbf{Similarity level} & 1 \\
\hline
\end{tabular}
\end{table}

\noindent Example three:
\begin{table}[h!]
\begin{tabular}{|@{\ }r@{\ \  }p{0.72\columnwidth}|}
\hline
\textbf{Sentence 1:} & \emph{A large portion of modern music an art has been inspired by marijuana}\\
\textbf{Sentence 2:} & \emph{Used as a medicine for its positive effects}\\
\textbf{Similarity level} & 4 \\
\hline
\end{tabular}
\end{table}
