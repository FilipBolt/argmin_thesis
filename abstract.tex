\thispagestyle{empty}

\section*{Summary}

This thesis focuses on several tasks in argumentation mining
of internet claims. Argumentation mining studies argumentation
	extraction, structuring, and analysis from text. 
Argumentation analysis is done in numerous applications, such as
the analysis of scientific papers, policy decision making, or
language learning through critical thinking development. 
With the increase of internet use, 
internet discussions are becoming a valuable source of 
argumentation. Claims constitute the building blocks of
argumentation. 

This research proposes methods for mining topic-specific argumentative claim
analysis in internet discussions.  A framework and support for claim extraction
and structuring and analysis is proposed.  First, three preliminary research studies
based on unstructured approaches to argumentation mining are conducted.  Then,
based on the results of the preliminary research studies, claim structuring is
proposed.  Claims are structured using a two-level ontology: an upper
ontology and a domain ontology. The upper ontology is used to describe
claim patterns.  The domain ontology models domain-specific concepts.
Structuring claims allows for higher quality claim analysis by relaxing
the problem of language variance and allows for deriving implicit claims.
Supervised machine learning methods are proposed to detect and structure
claims from internet discussions.  A method for claim analysis is
proposed to analyze implicit claims of internet discussion participants. 

\vspace{1cm}
\textbf{Keywords}:  
argumentation mining, natural language processing, formal
knowledge representation, machine learning, structured prediction
